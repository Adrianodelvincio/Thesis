
\maketitle
\begin{abstract}
\commento{Short introduction (1 page of text), an overview on the entire thesis.}
 
The Mainz Radius Experiment (MREX) is an experimental campaign with the aim of determining fundamental properties of the equation of state (EOS) of nuclear matter. The equation of state contains all the thermodynamic quantities of a system of nucleons, as energy, pressure, temperature, density, and asymmetry between the number of neutrons protons in nuclear-matter. An important parameter, poorly-known at the state of current knowledge, is the slope of the symmetry energy at saturation density $L$, which quantifies the dependencies of the energy per nucleon associated with the changes in neutron-proton asymmetry. This key component controls how the energy of a system of nucleons change when there is a difference in the number of proton and neutrons. It is also an essential element for the determination of the radius of neutron stars, whose description is still determined by the EOS, despite being many order of magnitude higher than the physical dimensions of the nuclei.
The slope of the symmetry energy $L$ is strongly correlated to a characteristic shown by heavy nuclei, the neutron-skin thickness, that is the difference between the  spacial distribution radius $R$ of the neutrons and protons. Nowadays it is well-known, thanks to various nuclear physics experiments, that the neutrons of a nucleus tend to accumulate at a larger radius, forming a neutral thin layer around atomic nuclei. This peculiar characteristic is known in literature as neutron-skin thickness. The experimental measurement of this quantity is the main method to estimate the value of $L$, which is used as an input to many theoretical models of neutron stars.  
The MREX is focused on the determination of the neutron skin thickness of $^{208}Pb$ from parity-violating experiments (PV) performed at the future MESA electron accelerator, that is currently under construction and will be located in Mainz. The parity-violating experiments, where longitudinal polarized electrons scatter from a fixed target at a single value of momentum transfer, consist in the determination of the cross section asymmetry $A_{pv} = \frac{\sigma^{+} - \sigma^{-}}{\sigma^{+} + \sigma^{-}}$ related to the different longitudinal polarization state of the beam. The parity-violating electron scattering is a valid probe to determine the neutron-skin thickness, because it is highly sensitive to the neutron distribution due to the larger coupling of the $Z^{0}$ boson to the weak charge $Q_{W}$ of neutrons, which is approximately $-0.99$ per neutron, while that of the proton is $0.07$. 
In this context, it is necessary to determine one of the possible background sources for the PV experiments, known as beam normal single spin asymmetry $A_{n}$, or transverse asymmetry. The asymmetry $A_{n}$, that concerns transversely polarized electrons, comes from the interference between two Feynman diagrams whith one or two virtual photons are exchanged, giving a contribution of the order of $20 \, ppm$. Because the values of $A_{n}$ are typical higher that $A_{pv}$, the presence of a small transverse electron polarization component could produce an effect that is of the same order of magnitude of the $A_{pv}$.
The work of this thesis focuses on the measurement of the transverse asymmetry $A_{n}$ carried out at the Mainz microtron accelerator (MAMI) on a $^{12}C$ target. The $^{12}C$ target is particularly suited for studying and testing the electronics systems and detectors that will be employed in the next phase of the MREX experiment, the determination of $A_{n}$ for $^{208}Pb$. The measurement consists in the determination of $A_{n}$ using two Cherenkov detectors made of fused-silica materials coupled to 3 and 8 photo-multiplier tubes. The two detectors have been tested in the laboratory, together with the new electronics for the data read-out, that consist in the NINO-asic board with which the impulse signals coming from the detectors are acquired. The beam parameters, as the transverse position of the beam, the scattering angles, and the current intensity and energy are determined with particular accuracy because their variation over time can result in effects that overlap with $A_{n}$. This required the development of a new analysis program, processing the raw data to extract the beam parameters relevant for the analysis, and separating the contributions of the false asymmetries from $A_{n}$. 
The work consisted in a first part dedicated to the calibration of the monitors, to measure the parameters of the beam. The second part was focused on the analysis of the data collected during the beam time, removing the outliers, identifying possible errors and isolating the contribution of false asymmetries.
$A_{n}$ has been measured for electron-carbon scattering at a two fixed angles ($\theta_{B} = -22.5 ^{\circ}$, $\theta_{A} = 22.5 ^{\circ}$) corresponding to a transfer momentum of $Q^{2} = \SI{0.04}{\giga \electronvolt \squared}$. The measured values are: $A_{B} = -21 \pm 5 \, (stat) \, ppm$ for detector B and $23.1 \pm 1.7 \,(stat)$ ppm for detector A. The different sign of the two measurements is in agreement with the opposite kinematic, and the two measurements are compatible within 1 $\sigma$, and in agreement with the previous measurements performed at MAMI. The results obtained confirm the capabilities of the electronic systems and components used during the experiments and are encouraging in anticipation of the next measurement of the transverse asymmetry for lead.
\end{abstract}
\newpage




\chapter*{Organization of Contents}

This thesis can be dived in two part: the first part is dedicated to the description and motivations regarding the MREX  experiment and the description of the MAMI accelerator, where much of the work was done. The second part is focused on the analysis of the data acquired during the beam time. A list of the chapters with a brief explanation of the contents follows:

\begin{itemize}
\item \textbf{chapter 1}: In this chapter the physics and motivation of the MREX experiment are presented. The Equation of state for nuclear matter is described, with particular attention on the symmetry energy $S$. Following we discuss the relevance of this parameter in many fields of physics, from Neutron stars to nuclear physics, and the experimental effort to measure this quantity from the determination of the Neutron skin thickness of $^{208}Pb$ 
\item \textbf{chapter 3} Description of MAMI experimental setup and beam characteristics. Explanation of beam measurement equipment and the interface with the electronics developed for the MREX experiment.
\end{itemize}

The result obtained are discussed and compared with the other measurements performed by different collaborations. Finally the confront with the theoretical prediction discussed in chapter 2 is done.