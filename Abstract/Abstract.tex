
\maketitle
\begin{abstract}
\commento{Short introduction (1 page of text), an overview on the entire thesis.}
 
The Mainz Radius Experiment (MREX) is an experimental campaign with the aim of determining fundamental properties of the equation of state (EOS) of nuclear matter. The equation of state is the fundamental quantity that contains all the thermodynamic properties of a system of nucleons and describes the relation between energy, pressure, temperature, density and asymmetry between the number of neutron and the number of proton in nuclear-matter.
 An important parameter, poorly-known at the state of current knowledge, is the slope of the symmetry energy at saturation density $L$, which quantifies the dependencies of the energy per nucleon associated with the changes in neutron-proton asymmetry. This key component controls how the energy of a system of nucleons change whether there is a difference in the number of proton and neutrons and it is also an underlying contribution to the determination of neutron stars radius whose physical description, despite being many order of magnitude higher than the physical dimensions of the nuclei, is still determined by the EOS, too.
The slope of the symmetry energy $L$ is strongly correlated to a characteristic shown by heavy nuclei, the neutron-skin thickness, that is the difference between the  spacial distribution radius $R$ of the neutrons and protons. Nowadays it is well-known from various nuclear physics experiments that the neutrons of a nucleus tends to locate more externally, forming a neutral thin layer around atomic nuclei. This peculiar characteristic shown by heavy nuclei is known in literature as neutron-skin thickness. The experimental measurement of this quantity is the main method to estimate the value of $L$, which is used as an input of many existing theoretical models of neutron stars.  
The MREX is focused on the determination of the neutron skin thickness of $^{208}Pb$ from parity-violating experiments (PV) performed at the future MESA electron accelerator, that is currently under construction and will be located in Mainz. The parity-violating experiments, where longitudinal polarized electrons scatter from a fixed target at a single value of momentum transfer, consist on the determination of the asymmetry $A_{p}$ in the number of scattered electrons due to the different longitudinal polarization of the beam, and are valid probe to determine the neutron-skin thickness of heavy-nuclei. 
In this context, it is necessary to determine one of the possible background sources for the PV experiments, known as beam normal single spin asymmetry, or transverse asymmetry, which consists in a small transverse electron polarization component which produce an effect that is of the same order of magnitude of the $A_{pv}$. The transverse asymmetry $A_{n}$ comes from the interference between the diagrams in which one or two virtual photons are exchanged, their contribution is on the order of 20 \textit{ppm} part-per-million.
The work of this thesis focuses on the measurement of the transverse asymmetry $A_{n}$ carried out at the Mainz microtron accelerator (MAMI) on a $^{12}C$ target. The $^{12}C$ target is particularly suited for studying and testing the electronic systems and detectors that will be employed in the next phase of the MREX experiment, the determination of $A_{n}$ for $^{208}Pb$. The measure consist in the determination of $A_{n}$ using two detectors made of two cherenkov fused-silica materials coupled to 3 and 8 photo-multipliers tubes. The two detectors have been tested in the laboratory, together with the new electronic for the data read-out, that consist in the NINO-asic board with which the impulse signals coming from the detectors are acquired.
The data collected during the beam-time have been analyzed developing a new analysis program, which in summary take care of processing the raw-data to compute the beam parameters that are relevant for the analysis. The beam parameters, as the $X$, $Y$ position of the beam (with $Z$ parallel to the beam direction), $\theta_{x}$ and $\theta_{y}$ scattering angles, and the $I$,$E$ current intensity and energy are determined with particular accuracy because their variation over time can result in effects that overlap with $A_{n}$. The analysis deals with separating the contributions of these false asymmetries from $A_{n}$. 
The work of analysis consisted in a first part dedicated to the calibration of the monitors, to measure the parameters of the beam, and the two detectors. The second part was focused in the analysis of the data collected during the beam time, removing the outliers and identifying and possible errors and isolating the contribution of false asymmetries.
$A_{n}$ has been measured for $e^{-}$-$^{12}C$ scattering at a two fixed angles ($\theta_{B} = -22.5 ^{\circ}$, $\theta_{A} = 22.5 ^{\circ}$) corresponding to a transfer momentum of $Q^{2} = \SI{0.04}{\giga \electronvolt}$. The measured values are: $A_{B} = -21 \pm 5 \, (stat)$ ppm (part per million) for detector B and $23.1 \pm 1.7 \,(stat)$ ppm for detector A. The different sign of the two measures is in agreement with the opposite kinematic, and the two measure are compatible within 1 $\sigma$, and in agreement with the previous measurement performed at MAMI.
\end{abstract}
\newpage




\chapter*{Organization of Contents}

the work-flow of my thesis can be summarized in a first part (corresponding to chapter 1 and chapter 2) which describes in details the underlying physics and the motivation for the measurement of the PV scattering and $A_{n}$, and a second part focused on the hardware work and data-analysis which is summarized in the following step:
\begin{itemize}
\item Description of MAMI experimental setup and beam characteristics. Explanation of beam measurement equipment and the interface with the electronics developed for the MREX experiment.
\item Description and characterization of the detectors with test in the laboratory.
\item Electronic tests and characterization. 
\item Development of the analysis program, check of the program with simple monte-carlo simulation together with specific unit test.
\item Description of the calibration phase, data pre-selection and removal of the outliers.
\item Data-analysis for $^{12}C$ and extraction of the $A_{n}$. Raw estimation of the systematic effects of the measurement and determination of the rates with the lead target, in prevision of the future experiment with $^{208}Pb$.
\end{itemize}

The result obtained are discussed and compared with the other measurements performed by different collaborations. Finally the confront with the theoretical prediction discussed in chapter 2 is done.