
\maketitle
\begin{abstract}
 
The Mainz Radius Experiment (MREX) is an experimental campaign with the aim of determining fundamental properties of the equation of state (EOS) of nuclear matter. All of the thermodynamic properties of a system of nucleons, including energy, pressure, temperature, density, and the asymmetry between the number of neutrons and protons in nuclear-matter, are contained in the EOS. An important parameter, poorly-known at the state of current knowledge, is the slope of the symmetry energy at saturation density $L$, which quantifies the dependencies of the energy per nucleon associated with the changes in neutron-proton asymmetry. It is also an essential element for the determination of the radius of neutron star, whose description is still determined by the EOS, despite a difference of many order of magnitude with respect to the physical dimensions of the nuclei.
The slope of the symmetry energy $L$ is strongly correlated to a characteristic shown by heavy nuclei, the neutron-skin thickness, that is the difference between the spacial distribution radius $R$ of neutrons and protons. Nowadays it is well-known, thanks to various nuclear physics experiments, that the neutrons of a nucleus tend to accumulate at a larger radius, forming a neutral thin layer around atomic nuclei. This peculiar characteristic is known in literature as neutron-skin thickness. The experimental measurement of this quantity is the main method to estimate the value of $L$, which is used as an input to many theoretical models of neutron stars.  
MREX is focused on the determination of the neutron skin thickness of $^{208}Pb$ from parity-violating experiments (PV) performed at the future MESA electron accelerator in Mainz, that is currently under construction. The parity-violating experiments, where longitudinal polarized electrons scatter from a fixed target at a single value of momentum transfer, consist in the determination of the cross section asymmetry $A_{pv} = \frac{\sigma^{+} - \sigma^{-}}{\sigma^{+} + \sigma^{-}}$ related to the different longitudinal polarization state of the beam. The parity-violating electron scattering is a valid probe to determine the neutron-skin thickness, because it is highly sensitive to the neutron distribution due to the larger coupling of the $Z^{0}$ boson to the weak charge $Q_{W}$ of neutrons, which is approximately $-0.99$ per neutron, while that of the proton is $0.07$. 
In this context, it is necessary to determine one of the possible background sources for the PV experiments, known as beam normal single spin asymmetry $A_{n}$, or transverse asymmetry. The asymmetry $A_{n}$, that concerns transversely polarized electrons, comes from the interference between two Feynman diagrams where one or two virtual photons are exchanged, giving a contribution of the order of $20 \, ppm$. Because the values of $A_{n}$ are typical higher that $A_{pv}$, the presence of a small transverse electron polarization component could produce an effect that is of the same order of magnitude of the $A_{pv}$.
The work of this thesis focuses on the measurement of the transverse asymmetry $A_{n}$ carried out at the Mainz microtron accelerator (MAMI) on a $^{12}C$ target. The $^{12}C$ target is particularly suited for studying and testing the electronics systems and detectors that will be employed in the next phase of the MREX experiment, the determination of $A_{n}$ for $^{208}Pb$. The measurement consists in the determination of $A_{n}$ using two Cherenkov detectors made of fused-silica materials coupled to 3 and 8 photo-multiplier tubes. The two detectors have been tested in the laboratory, together with the new electronics for the data read-out, that consist in the NINO-asic board with which the impulse signals coming from the detectors are acquired. The beam parameters, as the transverse position of the beam, the scattering angles, and the current intensity and energy are determined with particular accuracy because their variation over time can result in effects that overlap with $A_{n}$. This required the development of a new analysis program, processing the raw data to extract the beam parameters relevant for the analysis, and separating the contributions of the false asymmetries from $A_{n}$. 
The work consisted in a first part dedicated to the calibration of the monitors, to measure the parameters of the beam. The second part was focused on the analysis of the data collected during the beam time, removing the outliers, identifying possible errors and isolating the contribution of false asymmetries.
$A_{n}$ has been measured for electron-carbon scattering at a two fixed angles ($\theta_{B} = -22.5 ^{\circ}$, $\theta_{A} = 22.5 ^{\circ}$) corresponding to a transfer momentum of $Q^{2} = \SI{0.04}{\giga \electronvolt \squared}$. The measured values are: $A_{B} = -21 \pm 5 \, (stat) \, ppm$ for detector B and $23.1 \pm 1.7 \,(stat) \, ppm$  for detector A. The different sign of the two measurements is in agreement with the opposite kinematic, and the two measurements are compatible within 1 $\sigma$, and in agreement with the previous measurements performed at MAMI. The results obtained confirm the capabilities of the electronic systems and components used during the experiments and are encouraging in anticipation of the next measurement of the transverse asymmetry for lead.
\end{abstract}
\newpage

\chapter*{Organization of Contents}

This thesis is divided into two parts: the first part is dedicated to the motivation behind the MREX experiment and the description of MAMI electron accelerator, where most of the work was carried out. The second part focuses on the analysis of the data collected during the experiment. The list below is an overview of each chapter and a brief explanation of its contents:

\begin{itemize}
\item \textbf{chapter 1}: Motivation of the MREX experiment for the measurement of the neutron skin thickness of $^{208}Pb$. The equation of state (EOS) for nuclear matter is given, paying special attention to the slope of the symmetry energy at saturation density $L$. This term relates the EOS for nuclear matter to the structure of the neutron star, and is directly related to the neutron star radius. Its determination is of great significance for distinguishing various existing theoretical EOS models, and also for the internal structure of neutron star. Following, the parity violating electron scattering (PV) on $^{208}Pb$ is discussed. The PV scattering is the main experimental method to extract the neutron skin thickness $\delta r_{np}$, which is correlated to $L$. At the end of this chapter, we briefly introduce the Beam Normal Single Spin Asymmetry (BNSSA), also known as transverse asymmetry, the subject of this thesis. The BNSSA represents the most important background process in parity violating experiments, the determination of which is mandatory for isolating possible systematic effects. 
\commento{\item \textbf{chapter 2}: This chapter focuses on the physical description of the transverse asymmetry. A theoretical model for determining the value of the transverse asymmetry, together with a formula including the dependence of the BNSSA on transfer momentum $Q$, Compton and charge form factor are presented. The chapter ends with a review of the past measurement of the BNSSA performed at MAMI.}  
\item \textbf{chapter 2} This chapter focuses on the MAMI electron acceleration and its structure, based on a cascade of microtrons, to accelerate the electrons. One section of the chapter is dedicated to the beam monitors, that measure the beam parameters using resonant electromagnetic cavities which are excited by the passage of the electrons. These beam parameters are necessary to estimate and remove the effect of beam fluctuations on the total asymmetry $A_{tot}$. Finally, we briefly introduce  the setup for measuring the transverse asymmetry, and describe the two Cherenkov detectors used.
\item \textbf{chapter 3}: In this chapter some simple detector tests performed in the laboratory are introduced, and the result of the calibration phase, which precedes the transverse asymmetry measurement, is discussed. The calibration involves measuring the parameters to convert the raw data from beam monitors into data with the correct physical units. After calibrating the monitors, we explain the auto-calibration procedure, required to measure the PMT offset, that later are used to correct the asymmetry measurements.
\item \textbf{chapter 4}: The analysis of data is the main topic of chapter 5. For various values of the beam current, the rates with a lead target are determined, and the time required to measure the transverse asymmetry with lead with an precision of $\simeq 2 \, ppm$ is calculated. Following that, we discuss the data selection and the fit procedure for carbon data.
\item \textbf{chapter 5}: The result of the analysis is reported. The transverse asymmetry is measured by each PMT individually, and the result are averaged to obtain a single $A_{n}$ measurement for each detector.  
\end{itemize}