\begin{abstract}
\commento{Short introduction (1 page of text), an overview on the entire thesis.}
 
The Mainz Radius Experiment (MREX) is an experimental campaign with the aim of determing foundamental properties of the equation of state (EOS) of nuclear matter. The equation of state describes the relation between the foundamental quantities as energy, pressure, temperature, density and neutron-proton asymmetry of nuclear-matter. An important parameters, poorly-known at the state of current knowledge, is the slope of the symmetry energy at saturation density $L$, which quantifies the dependencis of the energy per nucleon associated with the neutron-proton asymmetry changes. This key component of the EOS is also an underlying constribution to the neutron stars radius whose physical description, despite being many order of magnitude higher than the physical dimensions of the nucleus, is still given by the equation of state.
The slope of the symmetry energy at saturation density is strongly correlated to a characteristic shown by heavy nuclei, the neutron-skin thickness, whose experimental measurement is main ingredient which allows the determination of $L$.  The neutron-skin thickness $\delta R_{np}$ of heavy nuclei is the difference between the spacial distribution radious $R$ of the neutrons and protons. 
The MREX is focused on the determination of the neutron skin thickness of $^{208}Pb$ from parity-violating experiments (PV) performed at the future MESA electron accelerator, located in Mainz. The parity-violating experiments, where logitudinal polarized electrons scatter from a fixed target at a single value of momentum trasfer, are a valid probe to determine the neutron-skin thickness of heavy-nuclei.
In this context, it is necessary to determine one of the possible background sources for the PV experiments, known as beam normal single spin asymmetry, or trasnverse asymmetry, which consists in a small transverse electron polarization componet which produce an effect that is of the same order of magnitude of the desidered one. \smallskip
The work of this thesis focuses on the measurement of the transverse asymmetry ($A_{n}$) carried out at the Mainz microtron accelerator (MAMI) on a $^{12}C$ target. The $^{12}C$ target is particolarly suited for studying and testing the electronic systems and the detectors that will be employed in the next phase of the MREX expertiment, that will be the determination of $A_{n}$ for $^{208}Pb$. With these objectives, the workflow of my thesis can be summarized in a first part (corresponding to chapter 1 and chapter 2) which describes in details the underlying physics and the motivation for the measurement of the (PV) and the transverse asymmetry, and a second part which is summarized in the following step:

\begin{itemize}
\item Description of MAMI experimental setup and the beam characteristics. Explanation of beam measurement equipment and the interface with the electronics developed for the MREX experiment.
\item Description and characterization of the detectors with test in the laboratory.
\item Electronic tests and characterization. 
\item Development of the analysis program and check with simple monte carlo simulations and specific unit test of the main functions implemented.
\item Description of the calibration phase, data pre-selection and removal of the outliers.
\item Data-analysis for $^{12}C$ and extraction of the $A_{n}$. Raw exstimation of the systematic effects of the measurement and determination of the rates with the lead target, in prevision of the future experiment with $^{208}Pb$.
\end{itemize}

The result obtained are discussed and compared with the other measurements performed by different collaborations. Finally the final confront with the theoretical prediction is shown.


\end{abstract}


