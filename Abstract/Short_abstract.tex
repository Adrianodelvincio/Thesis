\documentclass[11pt]{article}
\usepackage[left=1.5cm,right=1.5cm,top=2cm,bottom=2cm]{geometry}
\usepackage[output-decimal-marker={.}]{siunitx}
\author{Adriano Del Vincio}
\date{\today}
\title{Commissioning and first data analysis of the Mainz Radius Experiment}

\begin{document}
\maketitle
\begin{abstract}
The Mainz Radius Experiment (MREX) is an experimental campaign with the aim of determining fundamental properties of the equation of state (EOS) of nuclear matter. An important parameter, poorly-known, is the slope of the symmetry energy L, which quantifies the dependencies of the energy per nucleon associated with the changes in neutron-proton asymmetry. L is strongly correlated to the neutron-skin thickness $\delta r_{np}$, that is the difference between the mean square radius of the neutron and proton distributions. The MREX will measure $\delta r_{np}$ for $^{208}Pb$ from parity-violating experiments (PV). The parity-violating experiments, where longitudinal polarized electrons scatter from a fixed target, consist in the determination of the cross section asymmetry $A_{pv} = \frac{\sigma^{+} - \sigma^{-}}{\sigma^{+} + \sigma^{-}}$ related to the different longitudinal beam polarization state. In this context, it is necessary to determine one of the possible background sources, the so-called transverse asymmetry $A_{n}$, that concerns transversely polarized electrons. $A_{n}$ comes from the interference between two Feynman diagrams where one or two virtual photons are exchanged. The work of this thesis focuses on the measurement of $A_{n}$ carried out at the MAMI accelerator on a $^{12}C$ target, particularly suited for testing the electronics systems that will be used to measure $A_{PV}$. $A_{n}$ has been measured using two Cherenkov detectors (A and B) made of fused-silica materials coupled to 3 and 8 photo-multiplier tubes. The two detectors have been tested in the laboratory, together with the electronics that consist in the NINO-asic board with which the detector signals were acquired. The work consisted in a calibration phase, needed to measure the beam parameters, and in the analysis of the data. $A_{n}$ has been measured for electron-carbon scattering at a two fixed angles ($\theta_{B} = -22.5 ^{\circ}$, $\theta_{A} = 22.5 ^{\circ}$) corresponding to a transfer momentum of $Q^{2} = \SI{0.04}{\giga \electronvolt \squared}$. The measured values are: $A_{B} = -21 \pm 5 \, (stat) \, ppm$ for detector B and $23.1 \pm 1.7 \,(stat) \, ppm$  for detector A.
\end{abstract}
\end{document}