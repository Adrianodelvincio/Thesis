\chapter{Test and beam time analysis} \label{analysis}

\begin{outline}[enumerate]
\1 development of the analysis program (description of the Levenberg-Marquardt-Algorithmus.
\1 testing the analysis program with montecarlo data.
\1 Test of the detectors in the Lab.
\1 Beam line description.
\1 Data Analysis
	\2 thresholds scan
	\2 Rates on $Pb^{208}$.
	\2 Beam related asymmetry correction.
	\2 $C^{12}$ Asymmetry.
\end{outline}

\section{Model for fitting the data}
Here I have to explain the model used for describing the data, so the problem of the false asymmetry induced by variations in beam position, angle, current and energy. Here is a good point to explain the De Brujin sequence for the polarity patterns

\section{Data tree}
Explain how we compute all the values for the data tree, the position of the beam on the target, the angle, the correlated-difference values...

\section{Detectors test}

{\bfseries Explain the test of the two detectors in the lab, how we select the threshold, the correlation of the pmts and coincidence to select the threshold. Mention also that we observed two knees in the plot of counts vs. attenuation.}

The Nino board, which digitizes the signal from the PMTS, has two parameters which can be used to select the internal threshold of the discriminator, to cut the low amplitude signals and can be adjusted changing the settings of the DAQ program. These two parameters are \textit{Threshold} and \textit{Attenuation}. \textit{Threshold} means directly the charge value necessary for an impulse certain shape to be accepted by the signal discriminators. However the "physical" threshold can be also modified changing the \textit{attenuation}. The relation between threshold and attenuation is not linear, but follows:

\begin{figure}[hbtp]
 \centering
 \includegraphics[scale=0.4]{Analysis/ThrvsAtt.png}
 \caption{Threshold dependence against attenuation}
 \end{figure}

For our purposes, we select a common threshold values for all the pmts, and we decided to change only the attenuation values.\\
Before the Beam time, some test of the two detector were performed, to check that the pmts were still working and that the new electronic was able to count properly the pulses and store the data. For this studies, we didn't have a radioactive sources that we could use, so we moved the two detectors in the workshop of the accelerator, using the cosmic ray as a probe.

\subsection{Rate of Cosmic rays}

Knowing that the expected number of event for cosmic rays is about $1 \frac{event}{\SI{}{\centi \meter\squared} \SI{}{\second}}$ we can compute the expected values for the number of events. We decided to take 1 minute long acquisition for both the two detectors, this leads to $70$ expected events for detector B  and  $100$ events for detector A.  


\section{Analysis}

One of the main goal for this experiment was to measure the well know trasverse asymmetry of $^{12}C$, already measured before, as a test for the new electronic system. Previous measurements of the Transverse asymmetry have been performed for a carbon target. For this beam-time, the two spektrometers were placed at an angle such that the $Q^{2}$ values of the scattered electron is:

\begin{flushleft}
\begin{align*}
& Spektrometer A : \qquad Q^{2} = \SI{0.041337}{\giga \electronvolt} \qquad \textnormal{without Cut} \\
& Spektrometer A : \qquad Q^{2} = \SI{0.0394513}{\giga \electronvolt} \qquad \textnormal{with Cut} \\
& Spektrometer B : \qquad Q^{2} = \SI{0.0404771}{\giga \electronvolt} \qquad \textnormal{without Cut} \\
& Spektrometer B : \qquad Q^{2} = \SI{0.0405843}{\giga \electronvolt} \qquad \textnormal{with Cut} 
\end{align*} 
\end{flushleft}

The $Q^{2}$ values is the same of the last measurement performed at MAMI, and is measured with and without rejecting the inelastic electrons. 

\subsection{Alignment of the scattering plane}

\subsection{Calibration of the VFCs monitors}
Maybe it's important to divide this sections in two different part: the first part where I explain the Vfc convert the input voltage signal to a digital signal. In the second part just mention how we tuned the resistences (for X,Y monitors directly at the output signal with the oscilloscope, while for I21 and I13 monitors we used the data, so I'm able to produce plots only for the second ones).

\subsection{Calibration of the PIMO monitors}

For the calibration of the X Y monitors, we used two target made by three carbon wires at a certain distance from each other, aligned horizontaly and vertically. The distance between the two center of the external wires is $ d_{horizontal} = \SI{2.38}{\milli \meter}$ for the target alligned horizontally and $d_{vertical} = \SI{2.33}{\milli \meter}$ for the other one.
When the horizontal wires target is cetered, we turn on the beam, and we take some data slowly changing the horizontal beam direction. The beam direction is changed by MAMI operators, varying the Magnetic field of the \textit{Wobbler 16} magnets (\ref{fig:BeamLine}):

\begin{figure}[hbtp]
\centering
\includegraphics[scale=0.4]{figures/XYMOCalibBeamLine.svg.pdf}
\caption{Beam line scheme.}
\label{fig:BeamLine}
\end{figure}


 Then we repeat the same procedure with the other target, for the vertical direction. We observe that the pmts counts increase to a maximum, that is reached when the beam spot is centered on the carbon wire, and then decrease until the next carbon wire is hit by the beam.\\
We plot the pmt data \textit{versus} the $X25,X21,Y25,Y21$, given in $\SI{}{\volt}$. 
Given that we know the real distance between the two external wires, we can obtain the correct scaling factors to calculate the X and Y position values ​​of the beam. To identify the three peaks in of the carbon target, we fit the data using a gaussian model (see \ref{fig:HorizontalCalibration}). The mean $\mu$ represents the center of the wire, given in $\SI{}{\volt}$.
Looking at the Beam line, we assume that the beam travels in a straight line. Let's consider the \textit{Wobbler 16} magnet the "$0$" of a coordinate system, with the $z$ axis pointing to the target (left direction in the beam scheme). The Beam parameters are measured by the Monitors $X/Y_{21}, X/Y_{25}$, which are located at some distance respect to the target. Suppose we are working only with the $Y_{25}$ monitor (the procedure is the same for the others). The Beam $y$ position is described by:

\begin{align*}
y_{beam} = m \cdot (z - z_{wobbler 16})
\end{align*}

In the scheme \ref{fig:BeamLine} we easily compute the distance between the $Y_{25}$ monitor and the \textit{wobbler 16} magnet, so we have the slope $m$. The Position on the target is given by $Y_{target} = m \cdot Z_{target}$. With these simple equations then:

\begin{equation}
c_{Y25} = \dfrac{d_{vertical} [\SI{}{\milli \meter}]}{ Y_{target}} 
\end{equation}

$c_{Y25}$ indicates the scaling factor of the monitor. With these values the Analysis program compute the correct beam position, and from that the incident angles in the $x,y$ directions, which are needed later for the analysis.

\begin{figure}[hbtp]
\centering
\includegraphics[width=0.4\textwidth]{Analysis/HorizontalCalibration.png}
\caption{•}
\label{fig:HorizontalCalibration}
\end{figure}

All this procedure can be easely checked if we plot now the $X$ and $Y$ position for the same two runs of data acquired with the wires. After placing the scaling factors obtained in the standard configuration file, we run the analysis another time and the physical values were computed \ref{fig:CheckHori}

\begin{figure}[hbtp]
\centering
\includegraphics[width=0.45\textwidth]{Analysis/XcheckB.png} 
\includegraphics[width=0.45\textwidth]{Analysis/XcheckA.png}
\caption{plot of the pmt Count against the physical values computed by the analysis program. Now the position of the three peaks correspond to the expected values measured for the target.}
\label{fig:CheckHori}
\end{figure}


\subsection{Current and ENMO monitors}

For the current monitors I13 and I21, we perform the calibration changing the current of the beam and observing the output values of the monitors (Voltage values). The we perform a fit (for the beam current, we used the nominal values that we communicate to MAMI, has the values for the x-axis).

\begin{figure}[hbtp]
\centering
\includegraphics[width = 0.45\textwidth]{Analysis/I13_Calibration.png}
\includegraphics[width = 0.45\textwidth]{Analysis/I21_Calibration.png} 
\caption{•}
\end{figure}

For the two monitors we are able to compute the offset and scale factor:

\begin{equation}
\begin{split}
I^{volt}_{13} = m_{13} \cdot I^{Nom}_{13} + q_{13}\\
c_{13} = \frac{1}{m} \qquad offset = -\frac{q_{13}}{m}
\end{split}
\end{equation}

The same formula for current monitor I21.

The Enmo calibration is performed in a different way from the other monitors. The polarity signal is sent to MAMI, and they produce a signal for the ENMO that somehow (need to investigate exactly how they do that) shows a difference between the first two subevents and the last two. This difference is equal (nominal) to $\SI{22.6}{\kilo \electronvolt}$. The idea now is to produce an histogram for the quantity $\delta E$ (with $E_{18}$ being the energy monitor):

\begin{equation*}
\delta E = \frac{E_{18}[2] + E_{18}[3]}{2} - \frac{E_{18}[0] + E_{18}[1]}{2} 
\end{equation*}

The data should be distributed with a peak around $\SI{22.6}{\kilo \electronvolt}$. To obtain the correct scaling factor for the values stored in the data tree we plot the voltage values mesured by the ENMO monitor.
3 runs of data where taken with different Beam current, to study the dependence of the measured quantity from the beam current. From the mean of the distribution it is possible to exstimate the scaling factor for the ENMO monitors, obtaining the physical quantity in the following way:

\begin{equation*}
C_{E18} = \frac{\SI{22.6}{\kilo \electronvolt}}{\overline{\delta E}}
\end{equation*}

\begin{figure}[hbtp]
\centering
\includegraphics[width = 0.45\textwidth]{Analysis/ENMOvoltage20.pdf}
\includegraphics[width = 0.45\textwidth]{Analysis/ENMOvoltage15.pdf} 
\caption{$\delta E$ for 20 $\SI{20}{\micro \ampere}$}
\end{figure}

Taking the average over $E_{18}$ voltage values, and using the formula above, we obtain the coefficient $C_{E18}$. To take care of the current depencende of the monitors, the scaling factor to be placed in the standard.config file is: $C_{E18} \overline{I}_{\mu A}$.
The calibration was performed taking three short acquisitions with different beam current : $\SI{20}{\micro \ampere}$, $\SI{15}{\micro \ampere}$ and a run without beam. 

\begin{figure}[hbtp]
\centering
\includegraphics[width = 0.5\textwidth]{Analysis/E18_Calibration.png}
\caption{Calibration of ENMO monitor}
\end{figure}

From this we obtain the value $scaling_{E18} = -1595.2$, to obtain the physical quantity from the analysis. As a final check the final histogram for the physical quantity is shown:

\begin{figure}[hbtp]
\centering
\includegraphics[width = 0.45\textwidth]{Analysis/ENMOCheck20.pdf}
\includegraphics[width = 0.45\textwidth]{Analysis/ENMOCheck15.pdf} 
\caption{Plot for the physical quantities computed in the data tree, for two different current of the beam (on the left $\SI{20}{\micro \ampere}$, $\SI{15}{\micro \ampere}$ on the right)}
\end{figure}



\subsection{Calibration of the pmts}

{\bfseries Here it's important to show the plots I made during the beam time. I have to mention the Leo tecniques for the correct interpretation of counts vs attenuation.}

During the beam time, several scans in attenuation were performed, before switching MAMI to produce the polarized beam, to choose the best working point for the PMTS of the detectors. The same procedure used in the laboratory was followed, starting from low attenuation and raising up the values. It's possible to get a simple model to describe the particular shape of the following plot taking into account simple assumptions about the type of electrical noise that affect the Nino board, and the \textit{pdf} of the signal produced by the PMTS.
The main assumption ("this is not a true assumption, Anselm has a plot of the digitalize charge that proves that") is that the signal amplitude, in $\SI{}{\milli \volt}$ collected by the Nino board is well described by a gaussian distribution, and for signal with low amplitude, we expect to be well described by an uniform distribution. Just to visualize, let's suppose that the distribution of the signal amplitude collected is of this type (\ref{fig:PDF}) (the following figure is just an example, the values ​​do not describe the data collected):

\begin{figure}[hbtp]
\centering
\includegraphics[width = 0.40\textwidth]{Analysis/distribution.pdf}
\caption{Example of the expected distribution of the PMTs output signal}
\label{fig:PDF}
\end{figure}

The probability for a signal to pass the selection is equal to the probability of being in above the threshold, that is the complementary cumulative of the gaussian distribution (probability of being in the right tail):

\begin{align*}
P(signal > thr) = 1 - \Phi(x) = \dfrac{1 - Erf(\dfrac{x_{thr} - x_{0}}{\sqrt{2} \sigma })}{2}
\end{align*}

Once we reach the uniform zone, the probability of an even being selected is proportional to the area of the rectangle, which increases linearly decreasing the threshold. Considering the normalization factor, it is straightfoward to fit the data with a model of this type:

\begin{equation}
\begin{split}
N(att) = N_{0} \cdot \dfrac{1 + Erf(\dfrac{x_{thr} - x_{0}}{\sqrt{2} \sigma })}{2} \qquad \text{if} \quad att < C  \\
N(att) = N(C) + m \cdot (att - C) \qquad \text{if} \quad att > C
\end{split}
\end{equation}

For the detector B we show the result assuming our model: 

\begin{figure}[hbtp]
\centering
\subfloat[][\emph{Attenuation scan for the pmts of detector B} \label{fig:FitAtt}]
{\includegraphics[width = 0.5\textwidth]{Analysis/Fit_attenuation.pdf}}
\subfloat[][\emph{Recostructed spectra for Detector B, notice that we have \textit{Attenuation} values in the x axis, therefore right and left are swapped with respect to the graph \ref{fig:PDF}.} \label{fig:Spectra}]
{\includegraphics[scale = 0.5]{Analysis/Bspectra.pdf}}
\end{figure}

From the fit we obtain three valus for the signal Peak, given in attenuaton units.
We can check the idea behind this, visualizing the pmt count in a different way. In this plot (\ref{fig:FitAtt}) there is a discrete differentiation of the data showed in (\ref{fig:Spectra}). This plot represents roughly the signal spectra of the signal. 
We can see that our assumption is not far from what we see, except for the fact that we are not able to identify the linear area. This is not very importat, sice we have to indenty properly a good point to select all the real signals from the scattered electrons, rejecting the noise. Furthermore, if we look at the plot (), we can understand this behaviour looking that the threshold does not scale linearly with changing the attenuation value, for high values of attenuation, the threshold falls quickly at zero.

 
From the NINO documentation, it's possible to confront these values with 

\newpage

\subsection{Rates on lead}

{\bfseries This section is straightforward. Basically I have to show the single plot of the pmts counts vs. beam current for lead target. However it's possible to do some preliminary studies, for example to calculate the time needed for measuring the asymmetry on lead with a certain error and maybe check from Mott cross section that the observed rate are fine.}

\begin{figure}[hbtp]
\centering
\includegraphics[width = 0.45\textwidth]{Analysis/Rates_on_lead.png}
\includegraphics[width = 0.45\textwidth]{Analysis/Rates_on_leadB.png}
\caption{Rates on lead Target, for Detector A (left)}
\end{figure}

\section{Asymmetry on Carbon}
.
\subsection{Autocalibration procedure}
.
\subsection{Data selection and Fit}

After all the calibration are performed, the analysis program is ready to produce the datafiles suitable to analyze the asymmetry data for Carbon. \\
The Data file that are produced from the binary files are simply files in txt format, where the data are stored in columns. Before proceeding with the linear fit, however, it is necessary to visualize the data to check that there are no anomalous behaviors. In fact the data can contain moments of loss of the beam burrent and sudden interruptions, loss of polarization of the beam and even setting errors by MAMI operators can affect the experiment. Carbon data were taken from November 2nd to 4th, and consist of $28$ runs, each $1$ hour long.
The first step is to observe the pmt counts and the current trend, in order to be able to identify sudden interruptions of the beam. Here we show the trend over time for the series runs: 

\begin{figure}[hbtp]
\centering
\subfloat[][\emph{Pmt count trend versus time.In black A0, in green B0} \label{fig::CountTrend}]{
\includegraphics[width = 0.45\textwidth]{Analysis/BeamExample.pdf}}
\subfloat[][\emph{Asymmetry trend for pmt A0} \label{fig::AsymTrend}]{
\includegraphics[width = 0.45\textwidth]{Analysis/AsymmetryTrend.pdf}}
\end{figure}

This plot show that after 10 h of data acquisition the Pmt counts (\ref{fig::CountTrend}) dropped rapidly. If we show the current trend over the time (\ref{fig::CurrentTrend}) we do not see a corresponding decrease in beam intensity. Also the $x,y$ position (\ref{fig::PositionTrend}) and the energy monitor of the beam do not show a strange behavior. So we reject the possibility that at some point one of the stabilization device of the beam stop working. One of the possibilities is that something happen inside the two spectrometers. From the pmt counts we can suggest that the scattered electrons were not hitting our fused-silica detector,
in fact for pmt B0 we observe roughly 0 counts, and for pmt A0 $20000$ counts, which is compatible with the noisy offset of that pmt. This consideration strongly suggest as to reject those data from the analysis.\\
Apart from this, we observe in (\ref{analysis}) two steepy variations of the asymmetry around $13.5 h$ and $19 h$. For these data is quite simple to explain why should be rejected: we observe the same steepy variations also for the positions, so in this case the beam was not correctly aligned to the target.

\begin{figure}[hbtp]
\centering
\subfloat[][\emph{Current trend over time.} \label{fig::CurrentTrend}]{
\includegraphics[width = 0.45 \textwidth]{Analysis/CurrentTrend.pdf}}
\subfloat[][\emph{$X,Y$ beam position over time} \label{fig::PositionTrend}]{
\includegraphics[width = 0.45\textwidth]{Analysis/XYtrend.pdf} }
\end{figure}

After this first data selection, we check the correlated-difference data, which will be later used as variables for the fit. We produced histograms for all this quantites, that we remember are computed by the analysis program with the formula:

\begin{align*}
\delta x =  \frac{(x_{up,1} + x_{up,2})}{2} - \frac{(x_{down,1} + x_{down,2})}{2}
\end{align*}

We expect, if the stabilization of the beam is correctly set, that those quantities are gaussian distributed. A maximum likelihood fit with a Gaussian model is performed for each beam parameter, with Likelihood Ratio for the goodness of fit (gof). The statistic for the gof is:

\begin{align*}
\lambda(x) = 2 \sum_{i}^{n} (f_{i}(\mu) - k_{i}(1 + log(\frac{f_{i}(\mu)}{k_{i}})))
\end{align*}  

$\mu$ are the parameters of the \textit{pdf}, that are mean and variance of the gaussian, and $f_{i}(\mu)$ and $k_{i}$ are respectively the expected value for the i-nth bin and the observed value for the i-nth bin.

\begin{figure}
\centering 
\subfloat[][\emph{$X$ position correlated difference}]{
\includegraphics[width = 0.45\textwidth]{Analysis/X.pdf}} 
\subfloat[][\emph{$Y$ position correlated difference}]{
\includegraphics[width = 0.45\textwidth]{Analysis/Y.pdf}}\\
\subfloat[][\emph{$\theta_{x}$ correlated difference}]{
\includegraphics[width = 0.45\textwidth]{Analysis/Xp.pdf}}
\subfloat[][\emph{$\theta_{y}$ correlated difference}]{
\includegraphics[width = 0.45\textwidth]{Analysis/Yp.pdf}}\\
\subfloat[][\emph{Energy correlated difference}]{
\includegraphics[width = 0.45\textwidth]{Analysis/ENMO.pdf}}
\end{figure}

Looking at the histograms, we see that the beam correlated-difference values are centered around zero, the result are reported below:

\begin{center}
\begin{tabular}{|c|c|c|c|c|c|}
\hline 
\rule[-1ex]{0pt}{2.5ex} & $X [\mu m]$ & $Y[\mu m]$ & $Xp [\mu rad]$ & $Yp [\mu rad]$ & $E [eV]$ \\ 
\hline 
\rule[-1ex]{0pt}{2.5ex} $\mu$ & $1.31 \cdot 10^{-3}$ & $2.4 \cdot 10^{-4}$ & $3.2 \cdot 10^{-8} $ & $3.6 \cdot 10^{-9}$ & $0.0013$ \\ 
\hline 
\rule[-1ex]{0pt}{2.5ex} $sigma$ & $3.7 \cdot 10^{-1}$ & $2.9 \cdot 10^{-2}$ & $ 1.9 \cdot 10^{-5} $ & $6.5 \cdot 10^{-6}$ & $0.38$ \\ 
\hline 
\end{tabular} 
\end{center}

these results are encouraging, the stabilization of the beam has meant that only small differences are observed within an event for each monitor. This means that, unless the values for the false asymmetries are not very large, the effect of the beam variation are small.

When all the calibrations are performed, it is possible to proceed to generate the datafiles fot the fit program.

For a better visualization of the data, especially to observe the dependence of the asymmetry on the Beam parameters measured, it is useful to take the average asymmetry at regular intervals. From the raw plots of the asymmetries (see below \ref{fig:asyvsparam}), it is clear that the statistical error associated to the asymmetry is the main one, and it's not possible to identify a linear dependence.

\begin{figure}[hbtp]
\centering
\includegraphics[width = 0.5\textwidth]{Analysis/Asym_vs_monitor.png}
\caption{Asymmetries vs. Beam parameters}
\label{fig:asyvsparam}
\end{figure}

In some plots (X,Y and I) we can identify equally spaced cluster of data. So we decide to compute the averaged asymmetry for each cluster we were able to identify. 
In the following plots we decided to apply some cuts to the data, selecting only the events where the values of all the monitor are less then 3 standard deviation far from the mean. In all the figures the asymmetries are multiplied by a factor of $1e6$ to have the result in ppm (so each y-axis is in ppm). 

\begin{align*}
(x_{monitor} - \overline{x}_{monitor}) \leq 3 \cdot \sigma_{X}
\end{align*}

For each monitor we use the \textit{curve fit} function of the python library \text{scipy} to fit the data. Each Beam parameter is treated separately now, so in principle we are ignoring possible effect of correlation between the X-values (however, from the correlation matrix    \textit{write that somewhere} the effects are negligible)  

\newpage
\begin{figure}[hbtp]
\centering
\subfloat[][\emph{Asymmetries [ppm] vs X position [$mm$]}]
	{\includegraphics[width = 0.45\textwidth]{Analysis/clustering_XB.pdf}} \quad
\subfloat[][\emph{Asymmetries [ppm] vs Y position[$mm$]}]
	{\includegraphics[width = 0.45\textwidth]{Analysis/clustering_YB.pdf}} \\
\subfloat[][\emph{Asymmetries [ppm] vs $\theta_{x}$ angle [$rad$]}]
	{\includegraphics[width = 0.45\textwidth]{Analysis/clustering_XpB.pdf}} \quad
\subfloat[][\emph{Asymmetries [ppm] vs $\theta_{y}$ angle [$rad$]}]
	{\includegraphics[width = 0.45\textwidth]{Analysis/clustering_YpB.pdf}} \\
\subfloat[][\emph{Asymmetries [ppm] vs E [$keV$]}]
	{\includegraphics[width = 0.45\textwidth]{Analysis/clustering_EB.pdf}} \quad
\subfloat[][\emph{Asymmetries [ppm] vs I current [arb.unit]}]
	{\includegraphics[width = 0.45\textwidth]{Analysis/clustering_IB.pdf}} \quad
\label{fig:averageAsym}
\end{figure}

\begin{figure}[hbtp]
\centering
\subfloat[][\emph{Asymmetries [ppm] vs X position [$mm$]}]
	{\includegraphics[width = 0.45\textwidth]{Analysis/clustering_XA.pdf}} \quad
\subfloat[][\emph{Asymmetries [ppm] vs Y position[$mm$]}]
	{\includegraphics[width = 0.45\textwidth]{Analysis/clustering_YA.pdf}} \\
\subfloat[][\emph{Asymmetries [ppm] vs $\theta_{x}$ angle [$rad$]}]
	{\includegraphics[width = 0.45\textwidth]{Analysis/clustering_XpA.pdf}} \quad
%\subfloat[][\emph{Asymmetries [ppm] vs $\theta_{y}$ angle [$rad$]}]
%	{\includegraphics[width = 0.45\textwidth]{Analysis/clustering_YpB.pdf}} \\
%\subfloat[][\emph{Asymmetries [ppm] vs E [$keV$]}]
%	{\includegraphics[width = 0.45\textwidth]{Analysis/clustering_EB.pdf}} \quad
%\subfloat[][\emph{Asymmetries [ppm] vs I current [arb.unit]}]
%	{\includegraphics[width = 0.45\textwidth]{Analysis/clustering_IB.pdf}} \quad
\label{fig:averageAsym2}
\end{figure}


The error of each point is computed exploiting the same formula defined above (theory section; $N_{A/B}$ averaged pmt counts for each subevents and $n$ number of event in each interval):

\begin{align*}
\sigma_{Asym} = \dfrac{1}{\sqrt{2N_{A/B} \cdot n}}
\end{align*}

From this plots we can check if the linear model is good enough to descrive the depencence of out data and decide if the it should be useful to apply different cuts for certain beam parameters. We report now a first exstimation of the false asimmetries, later the data will be fitted without treat separately the beam parameters. From that we can learn the effects of the correlation between the data.

\begin{center}
\begin{tabular}{||c|c|c|c|c|}
\hline
pmt: & B0 & B1 & B2 & unit \\ 
\hline 
$\frac{dA}{dX} $  & $-66 \pm 37$ & $-65 \pm 37$ & $-80 \pm 47$ & $ \frac{ppm}{\SI{}{\micro \meter}}$\\ 
\hline 
$\frac{dA}{dY} $  & $34 \pm 219$  & $ 79 \pm 233$ & $ -213 \pm 219$ & $\frac{ppm}{\SI{}{\micro \meter} }$ \\ 
\hline 
$\frac{dA}{d\theta_{y}}$ & $-416 \pm 1181$  & $ -443 \pm 1227$ & $-1237 \pm 1130$ & $\frac{ppm}{\mu rad}$\\ 
\hline 
$\frac{dA}{d\theta_{x}}$ & $-672 \pm 297$ & $ -672 \pm 307 $ & $-845 \pm 380$ & $\frac{ppm}{\mu rad}$ \\ 
\hline 
$\frac{dA}{dE}$ & $ -0.004 \pm 0.016 $ & $-0.011 \pm 0.016 $ & $ - 0.044 \pm 0.018 $ & $\frac{ppm}{\SI{}{\kilo \electronvolt}}$\\ 
\hline
\end{tabular} 
\end{center}


\subsection{False asymmetries}
{\bfseries Seems that is possible to obtain rough estimates of the beam related asymmetries with the results from the fit. For Energy and position it's achievable, while for the angles it's quite hard (in principle sounds possible to perform an analytic calculation of the asymmetry related to the incident beam angle, however Anselm told me that quite often those results are in disagrement with the observed even in the sign!).}

Until now the values for the false asymmetries were threated as the parameters of the fit. In this section we will investigate how we can obtain another different exstimations, usefult to check the validity of all the process of analysis of the data.

For $\frac{dA}{dX}$ and $\frac{dA}{dY}$, we conceptually exploit the possibility of varying the position of the beam on the target, as we did during one of the calibration phases. Using the same \textit{wobbler 16} we asked MAMI to slowly change the beam position on the X and Y monitor. The change in position has the effect to modify the rates for the two detector, and from them it's possible to extract exstimate the two false asymmetries related to the beam position. Now we will see how the two quantities are related.
From the plot .. we see that the counts are scaling linearly with the beam position, so we assume that the $N$ are given by

\begin{align*}
N(x,...) = N_{0} + m \cdot (x - x_{0})
\end{align*}

it is clear that the linear model can't be alwaws good, at some point the electron will be deflected completely out of the detector, however for the small variation that the magnets are producing this can be assumed without risk. The we can try to compute the asymmetry for two different beam position $x_{1}$ and $x_{2}$:

\begin{equation}
\begin{split}
Asym = \dfrac{N(x_{1}) - N(x_{2})}{N(x_{1}) + N(x_{2})} = \dfrac{N_{0} + m \cdot (x_{1} - x_{0}) - N_{0} - m \cdot (x_{2} - x_{0})}{N(x_{1}) + N(x_{2})} =  \dfrac{m}{2 N_{0} + m \cdot (x_{1} +  x_{2}) - 2m x_{0}}(x_{1} -  x_{2})
\end{split}
\end{equation}

We can approximate the denominator deleting the term $ m \cdot (x_{1} +  x_{2})$ which should be small compared to $2N_{0}$. We end with:

\begin{equation}
Asym = \dfrac{m}{2N_{0} - 2mx_{0}}(x_{1} -  x_{2})
\end{equation}

The term in front of $(x_{1} - x_{2})$ can be compared to $\frac{dA}{dX}$. $x_{0}$ is arbitrary and can be set to 0. For $N_{0}$, the offset, we substitute the averaged value counts of each pmt for the polarized beam acquisitions ( we remind that the rate are collected during each $\SI{20}{\milli \second}$ time interval of each sub-event).\\
The data are reported in the table below:

\begin{center}
\begin{tabular}{|c|c|c|}
\hline 
Pmt & Detector A & Detector B \\ 
\hline 
pmt 0 & 84718 & 18925 \\ 
\hline 
pmt 1 & 96882 & 18815 \\ 
\hline 
pmt 2 & 80604 & 14807 \\ 
\hline 
pmt 3 & 65053 & \\ 
\hline 
pmt 4 & 45943 & \\ 
\hline 
pmt 5 & 46248 & \\ 
\hline 
pmt 6 & 37452 & \\ 
\hline 
pmt 7 & 25808 & \\ 
\hline 
\end{tabular} 
\end{center}
 
We report the values obtained with this new method ... 

\begin{figure}[hbtp]
\centering
\subfloat[][\emph{Plot for slow variation in $x$ direction for detector B.}]
{\includegraphics[scale=0.5]{Analysis/slowxVariation.pdf}}
\subfloat[][\emph{Plot for slow variation in $x$ direction for detector B.}]{\includegraphics[scale=0.5]{Analysis/slowyVariation.pdf} }
\end{figure}

We can investigate also the asymmetries related to the beam energy. For this one we can exploit the theoretical expression for the Mott cross-section, taking the derivative ...

\subsection{??Bootstrap??}

Although Anselm was against it, now seems possible to increase the precision of the mesurement with a procedure similar to a bootstrap. Instead of computing all the quantities inside a single event, it's possible to compute all the important quantities also between different events. In this scenario the statistics can be increased artificially as mutch as we want, with the same amount of data. Of course, it's also simple to abuse of this method, so we should restrict using only events next to each other. However seem reasonable and promising.

\subsection{??interval estimation??}




