\chapter{Test and beam time analysis} \label{analysis}

\begin{outline}[enumerate]
\1 development of the analysis program (description of the Levenberg-Marquardt-Algorithmus.
\1 testing the analysis program with montecarlo data.
\1 Test of the detectors in the Lab.
\1 Beam line description.
\1 Data Analysis
	\2 Rates on $Pb^{208}$.
	\2 Stabilization Monitors.
	\2 $C^{12}$ Asymmetry.
\end{outline}

\section{Model for fitting the data}
Here I have to explain the model used for describing the data, so the problem of the false asymmetry induced by variations in beam position, angle, current and energy. Here is a good point to explain the De Brujin sequence for the polarity patterns

\section{Data tree}
Explain how we compute all the values for the data tree, the position of the beam on the target, the angle, the correlated-difference values...

\section{Detectors test}
Explain the test of the two detectors in the lab, how we select the threshold, the correlation of the pmts and coincidence to select the threshold

\section{Analysis}

\subsection{Alignment of the scattering plane}

\subsection{Calibration of the VFCs monitors}

\subsection{Calibration of the PIMO monitors}

\subsection{Calibration of the pmts}

\subsection{Rates on lead}

\section{ $^{12}C$ asymmetry}

\subsection{least square fit}

\subsection{False asymmetries}

\subsection{??Boostrap??}

\subsection{??interval estimation??}




