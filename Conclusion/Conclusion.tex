\chapter{Result} \label{result}

\commento{\begin{itemize}
\item asymemtries on carbon.
\item expected rates on lead.
\item false asymmetriies result.
\item average of the asymmetries for the pmts.
\item confront with the theory.
\end{itemize}}

In this chapter we report the result obtained for the data-analysis. First we report the averaged asimmetries with and without subtracting the pmt offset. From the asymmetry results, we can compute the factor $c$ as the ratio between the final asymmetries with and without subtracting the offset. The values can be directly confronted with the ones defined in \ref{Autocalib}. We see a good agreement. All the values are in reported in ppm (part-per-million).

\begin{table}[!ht]
\centering
\subfloat[][\emph{Asimmetries, with offset not subtracted.} \label{table:NotCorrected}]{ 
\begin{tabular}{c|c|c}
\hline
 PMT   &   Average &   $\sigma$ \\
\hline
 B0    &    -21.64 &      7.9 \\
 B1    &    -19.96 &      7.9 \\
 B2    &    -24.91 &      8.9 \\
 A0    &     18.83 &      3.7 \\
 A1    &     15.92 &      3.5 \\
 A2    &     18.8  &      3.8 \\
 A3    &     18.54 &      4.2 \\
 A4    &     20.47 &      5   \\
 A5    &     22.61 &      5   \\
 A6    &     18.45 &      5.6 \\
 A7    &     18.41 &      6.7 \\
\hline
\end{tabular}} \qquad
\subfloat[][\emph{Asimmetries with offsets subtracted}\label{table:OffsetCorrected}]{
\begin{tabular}{lrr} 
\hline
 PMT   &   Average &   \sigma \\
\hline
 B0    &    -22.46 &      7.9 \\
 B1    &    -20.73 &      7.9 \\
 B2    &    -25.33 &      8.9 \\
 A0    &     24.25 &      3.7 \\
 A1    &     21.6  &      3.5 \\
 A2    &     24.25 &      3.8 \\
 A3    &     22.58 &      4.2 \\
 A4    &     23.53 &      5   \\
 A5    &     25.88 &      5   \\
 A6    &     20.86 &      5.6 \\
 A7    &     19.89 &      6.7 \\
\hline
\end{tabular}}
\qquad
\subfloat[][\emph{$c$ factor, as defined in \ref{eq:Systematic}} \label{table:Cfactor}]{
\begin{tabular}{c|c} 
\hline
 PMT   &        c \\
\hline
 B0    & 0.96 \\
 B1    & 0.96 \\
 B2    & 0.98 \\
 A0    & 0.78 \\
 A1    & 0.74 \\
 A2    & 0.78 \\
 A3    & 0.82 \\
 A4    & 0.87 \\
 A5    & 0.87 \\
 A6    & 0.88 \\
 A7    & 0.93 \\
\hline
\end{tabular}}
\caption{Averaged asymmetries over all the events. The values are corrected subtracting $\overline{A}_{I}$ and considering the effective polarization $p$ of the beam}
\end{table}

The asymmetries are shown with the errors in the following plot. The error are obtained with the formula:
\begin{align*}
\sigma = \sqrt{\frac{1}{2 N \cdot n}}
\end{align*}

To Obtain a final asymmetry for detector A and B, the asymmetries for each plot are averaged using the formula:

\begin{equation}
\overline{A_{n}} = \sum_{i = 0}^{n_{PMT}} \dfrac{ w_{i} A_{i}}{\sum_{i = 0}^{n_{PMT}} w_{i}}
\end{equation}

This is a weighted mean, and $w_{i} = \frac{1}{\sigma^{2}_{i}}$. This formula is applied to take care of the different statistical error for different pmts.

\begin{figure}[hbtp]
\centering
\includegraphics[width = 0.80\textwidth]{Analysis/Dataselection/FirstResult.pdf}
\caption{Plot of the asymmetries ordered by the pmt label, the result are the average event per event, corrected by the beam asymmetry current and for the polarization $p$ percentage.}
\end{figure}

The final result obtained, without any further analysis, are: 
\begin{itemize}
\item Asymmetry for detector A, $A_{A} =  23.1 \pm 1.6$ ppm.
\item Asymmetry for detector B, $A_{B} = -22.7 \pm 4.7$ ppm.
\end{itemize}


\chapter{Conclusion and outlook} \label{conclusion}

\commento{\begin{itemize}
\item outlook for the future experiments with lead.
\item mention the future experiment with Parity-violatin scattering.
\end{itemize}}