\chapter{Experimental setup} \label{experiment}

\begin{itemize}
\item description of MAMI, how the beam is produced, how the electrons are polarized.
\item description of A1.
\item description of beam stabilization, how the monitors measure the beam parameters.
\item Electronics description, DAQ system, VFC monitors.
\item Detectors A and B.
\end{itemize}

\section{First description of the experiment}

First description of the experiment, how we want to collect data, new picture of the kinematic of the experiment. Maybe here it's a good point to describe the structure of the event.

\section{Mami}
How Mami produces polarized electron and how the particle are accelerated (the way Mainz Mikroton is working is completely different from the other accelerators, so maybe this section will be too long).

\subsection{Polarized Beam}
Here a subsection to explain how the polarized electrons are produced. Important to mention the systematic error for the polarization mesurement (in our beam time we couldn't measure with Moller polarimeter, so this discussion is important for future experiment, however it's important to say something about it).

\section{A1 spectrometers hall}

Describing the A1 room, how the spectrometers are operating (+ figures), a picture of the target (+ figure) and the important parameters for the target, like thickness. Mention that we need the Wobbler magnet to change the hitting position of the beam to prevent the target from melting.

\section{Detectors, and Monitors}

\subsection{Detectors A and B}
Describe the two detectors we placed inside the spectrometers, the $Q^{2}$ for our mesurement. The way the counts are collected, so the expected signal for the Čerenkov detector. Explain also how we will use the old detectors of the two spektrometers to align the elastic scattering plane to our detectors.

\subsection{Monitors and stabilization}
Explain how the monitors for the beam parameters work. (this section could be long, however the way these parameters are measured is particular, so it's important to explain everything properly).

\subsection{Mott polarimenter}
Briefly explain how the Mott polarimeter work, for measuring the polarization of the beam.

\section{Electronics}
Short introduction about the old electronics setup and why a new versions is needed, then describe all the electronics used for our experiment:
\begin{itemize}
\item Nino board for collecting the data from the pmts
\item VFCs for collecting the data from X21,X25,Y21,Y25,ENMO,I21,I13
\item master board for collecting the monitors data/controlling the source
\item small boxes for switching from new electronic read-out to the old electronics read-out (spectrometers DAQ)
\end{itemize}

