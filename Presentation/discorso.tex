\documentclass[11pt,a4paper]{article}
\usepackage[utf8]{inputenc}
\usepackage{amsmath}
\usepackage{amsfonts}
\usepackage{amssymb}
\usepackage[left=2cm,right=2cm,top=2cm,bottom=2cm]{geometry}


\author{Adriano del Vincio}
\title{Discorso per la presentazione}
\usepackage{xcolor}
\begin{document}
\maketitle

\section{PARTE INTRODUTTIVA}
\begin{list}{\textbf{slide}}{}

\item 1, Salve, buon pomeriggio sono Adriano Del Vincio e presenterò adesso il lavoro della mia tesi dal titolo "Commissioning and First Data Analysis of the Mainz Radius Experiment". Si tratta di una campagna sperimentale che si svolge presso l'acceleratore di elettroni MAMI, dell'instituto di fisica nucleare di Mainz, in Germania, dove si svolto gran parte del lavoro sperimentale. La mia tesi è incentrata sulla prima fase dell'esperimento MREX, con una prima presa dati e successiva analisi.

\item 2, l'obiettivo dell'esperimento MREX è quello di misurare la neutron skin thickness del nucleo di piombo 208. La neutron skin thickness è una caratteristica di quei nuclei atomici che presentano un significativo eccesso di neutroni rispetto al numero di protoni. La definizione di neutron skin è data in equazione 1, si tratta della differenza tra i raggi rms delle distribuzioni spaziali di neutroni e protoni che constituiscono il nucleo atomico. Per i nuclei pesanti la distribuzione spaziale dei neutroni tende ad essere più estesa rispetto a quella dei protoni. Il piombo 208 è scelto perchè presenta un significativo eccesso di neutroni (44) oltre ad essere un nucleo stabile, che lo rendono un candidato ideale per questo tipo di esperimenti.

\item 3:

\textcolor{purple}{Vers.1} \\la neutron skin è legata all'equazione di stato della meteria nucleare, che descrive la termodinamica dei nuclei atomici. Nonostante una differenza di svariati ordini di grandezza rispetto alle dimensioni del nucleo, l'equazione di stato riveste un ruolo importante anche per quanto riguarda la struttura delle stelle di neutroni, ad esempio la determinazione del raggio. 
la neutron skin dei nuclei atomici è originata dalla competizione di due termini dell'equazione di stato: il termine di superficie e quello di simmetria. L'energia di simmetria è il termine che quantifica la variazione di energia di legame dovuta alla presenza di un'asimmetria tra il numero di protoni e neutroni. In particolare, i modelli teorici legano la neutron skin al valore della slope dell'energia di simmetria, spesso indicato con il simbolo di L maiuscolo. In figura vediamo le distribuzioni dei protoni (linea continua rossa) e neutroni (linea continua scura) come predette dal modello FSUGold. I modelli sono concordi nel predire una valore positivo della neutron skin.

\textcolor{purple}{Vers.2} \\
Le distribuzioni spaziali di neutroni (linea continua scura) e protoni (linea continua rossa) sono mostrate nella figura di questa slide. Queste distribuzioni sono determinate dall'equazione di stato della materia nucleare, che descrive la termodinamica dei nuclei atomici. L'equazione di stato riveste un ruolo importante anche per quanto riguarda la struttura delle stelle di neutroni, ed entra in gioco nella determinazione del raggio. La teoria è concorde nel predire che la distribuzione dei neutroni del piombo 208 è più estesa rispetto a quella dei protoni, perciò ci aspettiamo di misurare un valore positivo della neutron skin. la neutron skin thickness ha origine della competizione di due termini che compaiono nell'equazione di stato: il termine di superficie e l'energia di simmetria. Quest'ultimo è il termine che quantifica la variazione di energia di legame dovuta alla presenza di un'asimmetria tra il numero di protoni e neutroni. La misura di MREX sul piombo 208, permetterà di ricavare un parametro dell'equazione di stato, la slope dell symmetry energy, spesso indicata in letteratura con la lettera L maiuscolo che andremo ad approfondire nella prossima slide.
	
\item 4, In questa slide vediamo la definizione di L e il suo legame con le neutron skin. In equazione 2 ho riportato un modello estremamente semplificato di equazione di stato. Nel primo termine abbiamo l'energia per singolo nucleone epsilon, che dipende dalla densità di materia ro e quadraticamente dall'asimmetria protone neutrone alfa. L corrisponde al coefficiente dell'espansione al primo ordine nella densità dell'energia di simmetria S di rho. Nella figura in basso a destra vediamo meglio il collegamento tra L e R skin. Ogni punto rappresenta un modello teorico di equazione di stato. Vediamo che emerge una relazione lineare tra le due quantità. Si comprende quindi che un'accurata determinazione della neutron skin dà accesso al valore di L.

\item 5 (aggiuntivo), Brevemente, come abbiamo già menzionato, il valore di L entra in gioco nella determinazione del raggio delle stelle di neutroni. Le stelle di neutroni si sorreggono grazie alla pressione interna, che contrasta la pressione gravitazionale che tenderebbe al collasso. Un importante contributo alla pressione interna è dato dalla pressione di degenerazione dei neutroni, tale pressione, usando il nostro semplice modello, è direttamente proporzionale ad L, come mostrato in equazione 3. Da qui si evince legame tra L e il raggio Rns. Nella figura è mostrata l'ellisse di covarianza tra L e $R_{ns}$ per stelle di neutroni di 0.8 e 1.4 masse solari predette dal modello in didascalia. Questo grafico conferma il legame tra L e raggio della stella di neutroni

\item 5, dopo questa prima parte introduttiva sulla teoria, vediamo adesso i dettagli di MREX. MREX misurerà la neutron skin del piombo 208 attraverso il parity-violating scattering, ovvero l'asimmetria nello scattering elastico elettrone nucleo. Per spiegare meglio l'esperimento consiste in elettroni \textcolor{blue}{POLARIZZATI} longitudinalmente che scatterano da un target di piombo. L'asimmetria è quella tra la sezione d'urto degli elettroni right handed e left-handed, come in equazione 4. L'asimmetria nasce dall'interferenza dei due diagrammi di feynmann mostrati. Il primo diagramma è legato al fattore di forma elettrico del nucleo di piombo, noto con grande precisione. il secondo diagramma è legato al fattore di forma debole. Poichè la carica debole dei neutroni è molto maggiore rispetto a quella dei protoni, quest'ultimo è sensibile alla distribuzione spaziale dei neutroni. Attraverso il parity violating scattering si misura il fattore di forma debole del nucleo di piombo, con cui si ricava il raggio della distribuzione spaziale dei neutroni e quindi la neutron skin. Si tratta di un esperimento pianificato all'acceleratore MESA, attualmente in fase di costruzione.

La parity-violating asymmetry sul piombo 208 è dell'ordine di 0.6 parti per milione. Per misurare una tale piccolo valore, è necessario tenere sotto controllo tutti i possibili errori sistematici. Un errore sistematico importante può essere legato ad una componente transversa residua degli elettroni. 

\item 6, tale componente può dar luogo ad un'asimmetria, asimmetria transversa appunto, che si va a sommare alla parity-violating asymmetry. Nello schema in basso a sinistra ho riportato i due casi di asimmetria pv e asimmetria transversa. Il lavoro della mia tesi corrisponde alla misura dell'asimmetria trasversa sul nucleo di carbonio 12. Misura necessaria per testare la nuova elettronica, con coi in futuro sarà misurata l'asimmetria transversa per il piombo 208. L'asimmetria transversa è il risultato dell'interferenza di due diagrammi di feynmann che corrispondono allo scambio di 1 e 2 fotoni virtuali.

\item 7,  Per il nucleo di carbonio ci si aspetta un contributo dell'ordine di 20 parti per milione. l'asimmetria viene misurata secondo la formula 5, andando a calcolare l'asimmetria tra i conteggi dei detector per la differente polarizzazione del fascio. Nel grafico sono riportate le misure precedenti che sono state fatte a MAMI con una diversa elettronica di acquisizione dei dati.

\item 8, qui in questa slide ho riportato uno schema dell'esperimento MREX, il lavoro della mia tesi corrisponde alla prima fase: la misura dell'asimmetria trasversa su carbonio 12. Il lavoro della mia tesi è riassunto nell'elenco a sinistra, e consiste nello sviluppo del programma di analisi, la calibrazione dei monitor che misurano i parametri del fascio, l'analisi per estrarre il valore dell'asimmetria e la misura dei rates con un target di carbonio in previsione della fase successiva di MREX.
\end{list}

\section{DESCRIZIONE DI MAMI}
\begin{list}{\textbf{slide}}{}
\item

\item

\item

\item

\item

\item

\item

\item 16 Questo è lo schema generale dell'esperimento, non entriamo troppo in dettaglio. Gli elettroni scatterati vengono rivelati dai due rivelatori A e B che sono posti a theta e meno theta angolo azimutale. I segnali vengolo letti dalla nino board poi comunicati al computer A1 che genera i file binari. Questi sono processati dal programma di analisi che porta all'estrazione della transverse asimmetry. I dati dei beam monitor sono invece inviati alla master board, che poi li invia al computer in A1. La master board genera pure il segnale per la sequenza della polarizzazione, che invia a MAMI.
\end{list}



\section{Beam Time}
\begin{list}{\textbf{slide}}{}

\item 17, Vediamo adesso quali sono gli obiettivi dell'esperimento. Per primo testare la nuova elettronica, sviluppata per esperimenti con rate più bassi (dell'ordine delle centinaia di kilohertz per il piombo e del megahertz per il carbonio). La vecchia elettronica si basava sull'integrazione dei segnali delle pmt lungo una finestra temporale. Questo metodo è soggetto, a causa delle presenza di rumore, non è adatto alle misure con rate più bassi. L'altro obiettivo che si lega a quello di prima è la misura dell'asimmetria trasversa per il carbonio 12. Oltre a questo sono stati misurati i rates con un target di piombo. Questi ultimi sono utili per stimare la statistica da raccogliere per misurare l'asimmetria trasversa sul piombo con la precisione desiderata. Infine l'ultimo obiettivo è quello di acquisire più conoscenza sui possibili effetti sistematici dell'esperimento. Si cita in particolare il problema delle false asimmetrie, ovvero asimmetrie indotte dalla variazione dei parametri del fascio che possono alterare l'esito delle misure. Per questo motivo, è necessario calibrare i monitor, per misurare i parametri del fascio.

\item 18,

\item

\item 19, Adesso discutiamo la calibrazione dei monitor che misurano la posizione del punto di incidenza del fascio sul target. Questa calibrazione avviene attraverso l'utilizzo di un target speciale costituito da 3 fili di carbonio (ottenuti da delle banali matite). La posizione del fascio viene fatta variare in maniera controllata grazie all'utilizzo di un magnete. Analizzando i conteggi dei detectors si osservano i tre picchi che corrispondono ai tre fili. Poichè la distanza tra i due fili esterni è nota, si può ottenere il fattore di conversione che converte i valori delle letture in volt dei monitor in unità fisiche (micro metri). Questa procedura è ripetuta per tutti e 4 i monitor che misurano la posizione del fascio.

\item 20, Per ricavare la posizione di incidenza del fascio, occorre assumere che esso si muova di moto rettilineo seguendo l'equazione di una retta. Si può andare a ricavare i punti di intercetta della traiettoria del fascio imponendo nell'equazione della retta il passaggio per i valori misurati dai monitor. La posizione quindi è calcolata dal programma di analisi combinando i valori dei due monitor XY21 e XY25.

\item 21, Per la calibrazione dei monitor di corrente ed energia si utilizzano due procedure diverse: per la corrente, si chiede a MAMI di impostare la corrente del fascio ad un certo valore, e contemporaneamente si misura il segnale in volt dei monitor I21. Si esegue poi un fit tra i valori nominali della corrente e quelli misurati. Il coefficiente angolare e l'offset sono i parametri di conversione cercati, che vengono salvati nel programa di analisi. Per quanto riguarda l'energia, MAMI è in grado di produrre uno shift artificiale nel valore di energia del fascio correlato con la polarizzazione. Tale shift corrisponde a 22.6 kilo electronvolt. Dalla differenza dei segnali dell'energy monitor, si può ottenere il fattore di conversione.

\item 22,

\item 23, Adesso possiamo affrontare l'analisi. Menzioniamo subito la struttura dei dati, che sono divisi in eventi e sotto-eventi.  Un evento ha la durata di 80 milli secondi ed è diviso in 4 sotto-eventi della durata di 20 milli secondi. A ogni sotto evento corrisponde un particolare stato della polarizzazione. Per misurare l'asimmetria, la polarizzazione del fascio viene fatta variare seguendo una delle due sequenze che ho riportato in figura. Le due sequenze sono ripetute in maniera randomica. In questo modo si possono cancellare prossibili contributi dovuti a rumore periodico che possono alterare le misure (un esempio di rumore periodico è quello della rete elettrica di alimentazione, con una frequenza di 50 hertz). Per ogni evento si ha una misura dell'asimmetria tra i conteggi dei sotto-eventi con polarizzazione up e down.

\item 24, Per questo esperimento è stata raccolata una statistica pari a 1 milione di eventi. I valori misurati delle asimmetrie, come si può vedere in figura, sono distribuiti in maniera gaussiana. La varianza dell'asimmetria è pari a 1 fratto due volte i conteggi della pmt che si va a considerare. L'errore associato alla media di N asimmetrie scala come 1 sulla radice delle misure. L'errore statistico finale è dato quindi dalla formula 8. Se si sostituisce alla formula 650000 numero di eventi e 20000 conteggi, come per la PMT B0 del detector B si ottiene il valore di 8 parti per milione.

\item 25, Nell'analisi occorre considerare il contributo delle false asimmetrie. Poichè l'asimmetria da misurare è piccola, dell'ordine di 20 parti per milione, può accadere che fluttuazioni dei parametri del fascio, correlati con la polarizzazione, possono dare luogo a false asimmetrie. I parametri importanti sono la posizione x y di incidenza, la corrente I, l'energia E e gli angoli di incidenza theta x e theta y. Per considerare questi effetti, si è utilizzato il modello in equazione 9 per descrivere i dati, supponendo che l'asimmetria misurata dipenda linearmente dai vari parametri. Le quantità importanti che entrano dentro il modello sono le differenze correlate alla polarizzazione, definite in equazione 10.

\item 26,

\item 27, Veniamo adesso alla selezione dei dati
ogni punto del plot a destra corrisponde ad una misura della correlazione tra i conteggi dei detector e la polarizzazione del fascio, ogni punto è 1 ora di presa dati, e abbiamo 23 punti che corrispondono alle 23 ore di beam time effettivo. Si può osservare un blocco punti in cui la correlazione è compatibile con 0. Questi dati non vengono considerati in quanto si ha il sospetto che si sia verificata una perdita di polarizzazione del fascio. Questo purtroppo ha comportato la perdita di un terzo della statistica disponibile. In giallo è plottato il valore della polarizzazione come predetto da un semplice montecarlo.

\item 28, Dopo la selezione dei dati, rimozione di outliers, si presentano i risulati dell'analisi. In questo plot sono mostrate le asimmetrie misurate per ogni PMT dei due detector usati. le linee tratteggiate corrispondono al valore ottenuto dalla media di ogni asimmetria per i due detectors. I risultati ottenuti sono positivi per il detector A e negativi per il detector B, in accordo con i segni che ci si aspetta dalla cinematica, negativo per i B e positivo per A. I valori misurati sono tutti dell'ordine di 20 parti per milione, inoltre si evidenzia come cambiando il segno delle asimmetrie per il detector B, i valori sono compatibili con le misure del detector A. 

\item 29, Per combinare i risultati di ciascuna PMT, si è effettuata una media pesata sull'errore. Il risultato finale del lavoro di questa tesi è riportato in formula 11. Per il detector A abbiamo un'asimmetria di 23.1 parti per milione, e -21 per il detector B. Queste misure sono in buon accordo con le altre misure effettuate a MAMI sul carbonio, che sono riportate in formula (12). I risultati sono incoraggianti perchè dimostrano il buon funzionamento della nuova elettronica sviluppata e aprono la strada alla fase successiva dell'esperimento MREX. 
\end{list}
\end{document}