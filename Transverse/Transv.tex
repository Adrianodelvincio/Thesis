
\chapter{Transverse Asymmetry} \label{transv}

\begin{itemize}

\item Physics behind the $A_{n}$ asymmetry, dependence on $Q^{2}$, the formula $\frac{\sigma_{\uparrow} - \sigma_{\downarrow}}{\sigma_{\uparrow} + \sigma_{\downarrow}}$
\item state of the art of the Exp.
\item Model description: so scattering amplitude, theoretical prediction
\item Expected error $ \delta A_{t} $
\item open question: problems with lead, dependence of $E_{beam}$, dependence from Z, Z/A
\end{itemize}

\section{Description of the process}

Explain the scattering process we are studying (at least one figure to visualize the kinematics of the scattering). Mention the link between this process and time-reversal operator. Add two figures for elastic and inelastic scattering.

\subsection{Elastic scattering}

Write the amplitude for the elastic (how to manipulate expression, maybe in the appendix). 

\subsection{Inelastic scattering}

Explain how it's possible to compute the inelastic expression, what kind of approximations are used (optical theorem...) 

\subsection{Model description}
Present the theoretical formula for the Transverse asymmetry, and comment on energy, Z, Z/A depencencies

\section{State of the Experiment}

Write down the formula $\frac{\sigma_{\uparrow} - \sigma_{\downarrow}}{\sigma_{\uparrow} + \sigma_{\downarrow}}$. Hints at how to measure the Transverse asymmetry (remember to mention we have a polarized beam against a unpolarized target). Explain the expected error for the recostructed asymmetry. Furthemore talk about the last mesurements obtained by the other collaborations, an outlook of the current situation. Maybe add also how we proceed to measure the transverse asymmetry, so the structure of the event, polarities patterns...



