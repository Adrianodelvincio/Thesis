
\chapter{Physics motivation for neutron skin thickness measurement.} \label{intro}
\commento{
\begin{itemize}
\item explain neutron skin thickness.
\item connection to neutron stars radious, and neutron stars description.
\item Equation of state (EoS) for high density nuclear matter.
\item Parity-violating scattering experiment for extracting neutron skin thickness.
\item mention the weak form factor.
\item Transverse asymmetry as background for Parity-violating experiment. 
\item Mention the other experiment, like PREX, that measure zero $A_{n}$ for Lead.
\end{itemize}}

\section{Nuclear equation of state (EOS) and neutron skin thickness.}

\commento{In this section we have to explain what is the neutron skin thickness and why this parameter is related to the Equation of State for nuclear matter (in particular, the slope of the Symmetry energy in the semiempirical mass formula). Then, explain the parallelism between Neutron stars and Nuclear matter (the share the same EOS), and underline the relation between radius of the neutron stars and EOS.}

During the 30s of the last century, a considerable part of the scientific community was concentrated in the study of the structure of atomic nuclei. The discovery that every atoms has a positive charged nucleus dates back to 1908, with the famous Rutherford experiment, where alpha particles scatter from a thin gold foil. In the following years, especially with the birth of quantum mechanics in the second half of the 1920s, significant progress were made in the knowledge of atomic nuclei and their properties. In 1935, a significant contribution was given by Carl Friedrich von Weizsäcker, that proposed the semi-empirical mass formula, to approximate the mass of an atomic nucleus. Although some refinements have been made over the years, the general structure of the formula is the same today. 
The model proposed by Weizsäcker is the application of the liquid-drop model for nuclear matter. The Nucleus is described as drop of protons and neutrons, that are assumed to be incompressible, which are held together by a nuclear potential. The semi-empirical mass formula states that the mass of a nucleus is given by 

\begin{equation}
m = Zm_{p} + Nm_{n} - \frac{E_{B}(N,Z)}{c^{2}}
\end{equation}

An important terms is the binding energy $E_{B}$, that contains 5 parameters:

\begin{equation}
E_{B} = a_{V}A -  a_{s}A^{\frac{2}{3}} - a_{c}\dfrac{Z^{2}}{A^{\frac{1}{3}}} -a_{asym}\dfrac{(N - Z)^{2}}{A} + \delta(N,Z)
\end{equation}

The first two terms $a_{V},a_{s}$ are taken from the liquid drop model, and are the volume energy and the surface energy. The volume term represent the energy due to the interaction of each nucleon with the other nearby nucleons. This term is proportional to $A$, that is the number of nucleon of the nucleus, which is proportional to the volume, hence the name. The second term represent is the surface energy, and it is a correction to the volume energy. The volume energy assume that each nucleon interact with a constant number of nearby nucleons, but this is not true if we consider the external protons and neutrons, because they have less neighbors to interact with. This correction terms is then proportional to $A^{\frac{2}{3}}$, that is the also proportional to the surface area. 
The third term $a_{c}$ denote the binding energy correction due to the repulsion between protons. The fourth term is $a_{asym}$, the asymmetry term, and it is proportional to the asymmetry between neutrons and protons. The theoretical justification for this terms is due to the Pauli exclusion principle. Neutrons and protons are distinct type of particles, and occupy different quantum states. Because neutrons/protons are fermions, they can't occupy a state with the same quantum numbers, therefore higher energy states are progressively filled. If there is an asymmetry between neutrons and protons, for example the number of neutrons is greater than the number of protons, some neutrons will be in higher energy states respect to the protons. The imbalance between the nucleons causes the energy to be higher respect to the situation with the same number of protons and neutrons. 
The last term the pairing term, and describes the effect of spin coupling, and has positive/negative values for even or odd N,Z. 
We want to focus on the fact that the liquid-drop model has the underlying assumption that the nucleons are incompressible. Because of this it's well defined the concept of saturation density, the fact that the density, at first order, is almost constant and independent of mass number A.
In the context of neutron stars, it's more useful to take the thermodynamic limit in which the number of nucleons and Volume are taken to infinity. The binding energy per nucleons can be written as:

\begin{equation}
\epsilon (\rho_{0}, \alpha) = -\frac{E_{B}}{A} = -a_{V} + a_{asym} \dfrac{\rho_{n} - \rho_{p}}{\rho_{n} + \rho_{p}}
\end{equation}

However, 

\section{Parity-violating scattering experiment}

\commento{This section is for describing the way it's possible to extract the neutron skin thickness. Here I have to mention the weak form factor and the important fact that the neutrons are more important than the protons in the parity-violating scattering, because of the weak mixing angle.}

\section{Transverse asymmetry}

\commento{Here we have to introduce the aim of this thesis: the transverse asymmetry is a source of background for the parity-violating experiments. Furthermore the theory is not working well for some nuclei ($^{208}Pb$), so mention PREX paper about the last measurement on carbon and lead, the problem that they measure $0$ transverse asymmetry.}

\subsection{Motivation}
\commento{Here present all the motivation for this thesis, so the fact that we want to measure the rates on lead for the future experiment, test the new electronics, measure another time the trasverse asymmetry on $^{12}C$}

\subsection{Conventions used}
\commento{It could be useful, here, to have a subsection to explain the terminology for this thesis, to avoid misunderstanding.}