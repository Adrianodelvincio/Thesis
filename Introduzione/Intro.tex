
\chapter{Introduction} \label{intro} 

\begin{itemize}
\item explain neutron skin thickness.
\item connection to neutron stars radious, and neutron stars description.
\item Equation of state (EoS) for high density nuclear matter.
\item Parity-violating scattering experiment for extracting neutron skin thickness.
\item mention the weak form factor.
\item Transverse asymmetry as background for Parity-violating experiment. 
\item Mention the other experiment, like PREX, that measure zero $A_{n}$ for Lead.
\end{itemize}

\section{Neutron skin thickness and EOS}

In this section we have to explain what is the neutron skin thickness and why this parameter is related to the Equation of State for nuclear matter (in particular, the slope of the Symmetry energy in the semiempirical mass formula). Then, explain the parallelism between Neutron stars and Nuclear matter (the share the same EOS), and underline the relation between radius of the neutron stars and EOS.

\section{Parity-violating scattering experiment}

This section is for describing the way it's possible to extract the neutron skin thickness. Here I have to mention the weak form factor and the important fact that the neutrons are more important than the protons in the parity-violating scattering, because of the weak mixing angle.

\section{Transverse asymmetry}

Here we have to introduce the aim of this thesis: the transverse asymmetry is a source of background for the parity-violating experimets. Furthemore the theory is not working well for some nuclei ($^{208}Pb$), so mention PREX paper about the last mesurement on carbon and lead, the problem that they measure $0$ trasverse asymmetry.

\subsection{Motivation}
Here present all the motivation for this thesis, so the fact that we want to measure the rates on lead for the future experiment, test the new electronics, measure another time the trasverse asymmetry on $^{12}C$